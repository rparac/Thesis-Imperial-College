% Luke feedback: Wee need to handle that more general concepts are often much more frequent then it is described 
% For example - bird has feathers should exist in all cases but it is only described when the feathers are special.

\chapter{Conclusion}

This paper demonstrates that a concept bottleneck model can be combined with human-generated explanations to improve NN explainability.
In doing so, we have discovered the following findings:
\begin{enumerate}
    \item \textbf{Logic-based methods can effectively be used to tackle some NLP seq2seq problems}.
    Such an approach was crucial as the datasets were tiny, consisting of only around 100 examples per problem.
    The dataset size made it impossible to use state-of-the-art NLP techniques such as transformers.
    On the other hand, the solutions made by hand-crafting or learning a set of rules could achieve good performance despite the dataset size.
    However, the limitation is the lack of scalability of the underlying ILASP \cite{RefWorks:RefID:54-ilasp} system used to learn a set of rules.
    It was applied successfully on the generalisation task, while it was far off its theoretical optimality guarantee for the atomisation task.
    When we applied it successfully, it did produce a better solution than a hand-crafted set of rules.
    
    
    \item \textbf{The newly proposed CoDEx pipeline significantly outperforms its predecessor}. 
    We have replaced the extraction part of the original pipeline with the new atomisation and generalisation stages.
    A new set of concepts is much more informable and better at predicting the final label.
    There is also an indication that the concepts it produced are better, but a large-scale human study must be conducted to validate such results.
    However, the CoDEx module still has room for improvement, as it cannot account for concepts occurring in the video but not explicitly in the explanation.
    This limitation leads to poor concept precision values.
    In addition, the limitation makes applying the sequential training procedure challenging, which splits the training of concept and label prediction parts of the network.
    Sequential training would be preferred, given that the joint training may not truly use concepts to predict the results.
    We further show that the CoDEx pipelines, although well performing, do not extract a valuable set of concepts given a set of general image descriptions.
    So, the applicability scope of this method needs to be further investigated. 
    But, we believe it should work for any domain where the human-generated explanation is written to describe the final label.
    
    \item \textbf{The proposed logic-based classification produces high-performing and easily explainable solutions}.
    We show that the method can be applied to the concept bottleneck pipeline exceptionally well.
    It helped achieve much better results for the MLB-V2E dataset \cite{RefWorks:RefID:16-2021automatic} compared to the original concept-bottleneck models while being completely interpretable.
\end{enumerate}

\section{Future work}

There are several directions possible to take this project further. We highlight some of the ideas for doing so in this section. \\

The most pressing concern is the CoDEx sparsity.
The human-generated explanations do not capture all of the concepts present in the concept matrix explicitly.
For example, both \emph{The ball was caught by the outfielder} and \emph{The outfielder got the pitch} would be extracted as separate concepts.
These sentences could have been grouped with some combination of hyper-parameters. 
However, finding such values often had an undesirable knock-on effect.
For example, \emph{The ball was outside of the strike zone} and \emph{The ball was inside of the strike zone} would almost always be grouped before the first two sentences are. \\
% INSERT An Inference Model for Semantic Entailment in Natural Language DONE
To mitigate that issue, we might use models that solve the semantic entailment problem.
The semantic entailment task aims to determine whether some sentence S could be inferred by a sentence T.
Such information would allow us to group \emph{The ball was caught by the outfielder} and \emph{The outfielder caught the pitch} seamlessly \cite{RefWorks:RefID:85-salvo2005inference}.


Another possible direction could involve improving upon the video classification network.
We could enhance the current neural network architecture.
% INSERT Deep residual learning for image recognition DONE
It consists of image features extracted by ResNet \cite{RefWorks:RefID:86-he2016deep}, from which motion features are captured using 1D convolution.
Much better architectures do exist, such as the ones discussed in \ref{video-classification}.
We expect that an improved architecture should improve concept prediction performance, which in turn improves the final class prediction.

Comparing the performance with the original concept bottleneck \cite{RefWorks:RefID:35-koh2020concept} would also be beneficial.
This work extends upon the original idea by mining the concepts from explanations.
% INSERT The caltech-ucsd birds-200-2011 dataset DONE
The original work uses a human-engineered set of valuable concepts, which they applied to the CUB-birds dataset (X) \cite{RefWorks:RefID:69-wah2011caltech-ucsd}.
Comparing with the same dataset would give more insight into the generality of the concepts extracted by the CoDEx pipeline.


Moreover, the concept explanation quality should be evaluated using a human study, such as a Mechanical Turk analogous to \cite{RefWorks:RefID:16-2021automatic}.
Such a study would give more confidence regarding the quality of generated explanation.
In addition, we should also construct a human study interpreting the logic-based classification framework.

Finally, we could improve upon the atomisation procedure.
The atomisation procedure is the main culprit behind any concepts that are invalid sentences.
With the Jaccard score at around 0.5, the atomisation task has a lot of room for improvement. \\
% INSERT reference Exploring the limits of transfer learning with a unified text-to-text transformer. DONE
A possible approach for handling this issue may involve trying to solve this problem using a seq2seq transformer such as T5 \cite{RefWorks:RefID:88-raffel2020exploring}.
It would require greatly extending the existing dataset, but it would undoubtedly perform better with a sufficient number of examples.
