\chapter{Related Work}

% TODO: Alessandra feedback: Check the task we are doing is a form of semantic entailment, if it is pick 1/2 paper explaining it.

\section{Video Classification}

Even though the work on video classification is not closely related to this thesis's contributions, it should serve as a performance benchmark of the entire system.

There are a few popular datasets for video classification out there, such as YouTube-8M \cite{RefWorks:RefID:10-abu-el-haija2016youtube-8m:} and Kinetics \cite{RefWorks:RefID:9-carreira2017quo}.
The YouTube-8M \cite{RefWorks:RefID:10-abu-el-haija2016youtube-8m:} is an extensive benchmark for general video classification with around 8 million clips and 4800 visual entities.
Mao et al. \cite{RefWorks:RefID:4-mao2019hierarchical} achieved very high performance on the YouTube-8M dataset using a deep convolutional graph neural network and a multi-level feature extractor.
Kinetics-700 \cite{RefWorks:RefID:14-carreira2019short} dataset is the latest version of the popular Kinetics dataset, which contains 700 human action classes and around 650000 video clips.
The top-performing model at the dataset has been designed by Yan et al. which uses Multiview Transformers for Video Recognition. 
The core idea of this approach is to input multiple different "views" of an input video into a transformer model to achieve better results.
This approach resulted in a model with an accuracy of 82.2\% on the Kinetics-700 dataset with the top-5 accuracy of 95.7\%.

YouTube-8M and Kinetics datasets are somewhat different to the dataset the majority of the project will address.
The MLB-V2E sequences look very similar, given that all of the sequences start with a pitcher throwing a ball.

A dataset much more closely related to our problem is the MLB-YouTube \cite{RefWorks:RefID:3-piergiovanni2018fine-grained} dataset.
The segmented video classification part of this dataset is the base for the MLB-V2E \cite{RefWorks:RefID:16-2021automatic} dataset which also includes crowd-sourced explanations. 
The MLB-YouTube dataset is quite different from the alternatives discussed in this section because the scene is very similar in all videos, only a single camera view is used, and the activity itself is only different because of the movement of one person.
The MLB-YouTube paper validates the InceptionV3 \cite{RefWorks:RefID:49-szegedy2016rethinking} and I3D \cite{RefWorks:RefID:9-carreira2017quo} networks performance on the constructed datasets shown in the table below.
These results should serve as a guide for the results of this project.

\begin{center}
\begin{tabular}{ |p{2cm}|p{1cm} p{1cm} p{1cm} p{1cm} p{1cm} p{1cm} p{1cm} p{1cm}|  }
 \hline
 \multicolumn{9}{|c|}{Per-class average precision for the multi-class baseball classification} \\
 \hline
 Method & Ball & Strike & Swing & Hit & Foul & In Play & Bunt & Hit by Pitch \\
 \hline
 Random & 21.8 & 28.6 & 37.4 & 20.9 & 11.4 & 10.3 & 1.1 & 4.5 \\
 InceptionV3 & 66.9 & 93.9 & 90.3 & 90.9 & 60.7 & 89.7 & 12.4 & 29.2 \\
 I3D & 62.5 & 91.3 & 88.5 & 86.5 & 47.3 & 75.9 & 16.2 & 21.0 \\
 \hline
 
\end{tabular}
\label{inclusion-exclusion-rules}
\end{center}


\section{Definition of Concept in Other Settings}

There are different angles to the definition of concepts in various works.
As mentioned, this project defines a concept as a syntactic generalisation of an atomic sentence.
However, this approach is not common in other works which do concept extraction.


Formal concept analysis \cite{RefWorks:RefID:31-ganter2012formal} is a method for knowledge representation based on the lattice theory which can be used to extract hierarchical concepts.
It defines two fundamental notions: a formal context and a formal concept. 
Formal context K := (G, M, I) consists of a set of objects G, set of attributes M and relation I on (G, M).
If (g, m) $\in$ I, object g has attribute m.
Using these attributes, set A$' $ is defined containing all attributes common to the objects in a set of objects A.
Additionally, set B$' $ contains objects that have all attributes in the set of relations B.
From these definitions, formal concept of context (G, M, I) is defined as a pair (A, B) such that A $\subseteq$ G, B $\subseteq$ M, B$' $ = A and A$' $ = B.

The Formal Concept Analysis with text analysis is mainly used to construct the concept hierarchy, where a concept refers to a phrase containing two words.

For example, work by Cimiano et al. \cite{RefWorks:RefID:32-cimiano2005learning} extracts verb/subject, verb/object/ and verb/prepositional phrase as candidate concepts.
Moreover, the Formal Concept Analysis is used by Anoop et al. \cite{RefWorks:RefID:33-anoop2019extracting} to extract concepts with their relationships from unstructured text.
The authors manually extract stemmed noun phrases as key phrases and use indications such as "is-a" to generate a formal context table.
This table is then used to extract hierarchical concepts.

On the other hand, TaxoLearn \cite{RefWorks:RefID:34-dietz2012taxolearn} by Dietz et al. does not use Formal Concept Analysis but also extracts noun phrases as concepts. The relevance of noun phrases is checked by computing how often they appear in a particular context compared to other contexts. It also uses a hierarchical clustering algorithm to construct a taxonomy.

Moreover, approaches such as the one by Koh et al. \cite{RefWorks:RefID:35-koh2020concept} define concepts as properties that the image may have. These properties need to be provided beforehand.

Finally, work such as the one by Fan et al. \cite{RefWorks:RefID:50-fan2004semantic} use expert defined terms, such as \emph{Lecture Presentation for Gastrointestinal Surgery}, as semantic concepts. 


\section{Concept-Based Explanations for Images and Text}

Concept-based explanations for images and text attempts is a more straightforward related problem.
Koh et al. \cite{RefWorks:RefID:35-koh2020concept} predict a set of pre-labelled concepts which they use to make a final classification.
The pipeline by Koh et al. is similar to the pipeline that will be used in this project, just that the concepts are automatically extracted instead of provided.
Yeh et al. \cite{RefWorks:RefID:36-yeh2019completeness-aware} propose a method that automatically extracts sets of pixels from an image that represent valuable concepts.
Similarly, Ghorbani et al. \cite{RefWorks:RefID:37-ghorbani2019automatic} propose another method that automatically extracts a set of visual concepts which are meaningful to humans.
All outlined methods are based on visual concept extraction, unlike this project's concept extraction, which will mainly focus on the events/actions in the baseball video sequence.


\section{Video Explanation}

Another related problem is that of Video Explanations. This project may even explore it at a later stage within the context of a concept extraction pipeline.

One famous dataset for video understanding is the MSR-VTT dataset \cite{RefWorks:RefID:40-jun2016msr-vtt:}. It is a large scale dataset with more than 10000 video clips in 257 popular categories of a video search engine.
Each video is annotated with roughly 20 sentences. 
A few different approaches for tackling this issue are presented as viable options in the paper, such as 2D Convolutions, 3D Convolutions, and RNNs.
One of the most performant methods on this dataset is CLIP2TV \cite{RefWorks:RefID:41-gao2021clip2tv:}, which utilises transformer-based techniques for both video and text representation.


The authors of the MLB-V2E \cite{RefWorks:RefID:16-2021automatic} have also constructed the MSR-V2E \cite{RefWorks:RefID:16-2021automatic} dataset, which uses the clips made available by the MSR-VTT and provides new classification labels and explanations.
This dataset will likely be used for the evaluation of this project.

\section{Semantic Concept Video Classification}

Semantic Concept Video Classification attempts to predict a set of predefined concepts from a video.
Predicting a set of predefined concepts from a video is closely related to a part of the pipeline in this project, which indicates which extracted concepts occur before predicting the outcome. 

The work by Fan et al. \cite{RefWorks:RefID:50-fan2004semantic} tries to predict semantically defined concepts by medical experts.
The paper proposes using salient objects, which are visually-distinguishable video components that human semantics understands, to model these semantical concepts.
Examples of notions salient objects attempt to represent are a face, voice, or a lecture slide.
Despite its benefits, the semantics of salient objects is quite simple, and it does not model relationships that happen over multiple frames in a video.
Newer work by Fan et al. \cite{RefWorks:RefID:51-jianping2007incorporating} also incorporates the possibility of a hierarchical concept classification. The approach also predicts atomic video concepts before constructing the higher-level ones and existing concept ontology.

Assari et al. \cite{RefWorks:RefID:52-assari2014video} represent a video by the co-occurrence of the semantic concepts before applying a classifier.
The concepts in this work are also predefined, but they represent a set of events rather than a set of objects.


