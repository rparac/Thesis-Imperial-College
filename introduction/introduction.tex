\chapter{Introduction}

Natural language can be used as metadata for any sort of data.
To make it machine-readable and valuable for a task at hand, one needs to extract relevant pieces of information about it consistently.
If done successfully, given natural language explanations could even be used to explain unseen classification examples or improve the video classification performance.
In this project, a method for the automatic extraction of valuable information from language is studied as part of a bottleneck-based video classification pipeline.





\section{Problem Setup}

The project aims to design a general method for extracting valuable information from natural language.
The initial attempt at designing such a method intends to tackle a specific problem well before expanding its domain.
The problem this work is helping to solve is the problem of the classification of short baseball video sequences.
% TODO fix
In particular, it targets the MLB-V2E dataset  \cite{RefWorks:RefID:16-2021automatic}.
This dataset consists of the following:

 - A short video of a baseball sequence (\href{www.mlb.com}{MLB} (Major League Baseball) clips).
 
 - Label describing the outcome which occurred in it: strike, foul, ball, none, out.

 - Crowd-sourced brief explanation of why the outcome occurred.
 
 
\begin{center}
\begin{tabular}{ |p{1.5cm}|p{1.5cm}|p{9cm}|p{1.7cm}|  }
 \hline
 \multicolumn{4}{|c|}{Extract from MLB-V2E dataset} \\
 \hline
 id, $n$ & label, $l_n$ & explanation, $e_n$& video, $v_n$\\
 \hline
 1 & strike & The batter did not swing. The ball was in the strike zone. & N/A \\
 2 & foul & the batter hit the ball into the stands and it landed in foul territory & N/A \\
 3 & ball & The hitter didn’t swing. The ball was outside the strike zone. & N/A \\
 4 & none & The video did not load. & N/A \\
 5 & out & the batter hit the ball and it was caught by the fielder & N/A \\
 
 \hline
\end{tabular}
\end{center}


% TODO insert reference to the classifier %
Additionally, the problem is explored within the context of a concept bottleneck classifier. 
In that context, \textbf{concepts} are defined as syntactic generalisations of atomic sentences. 
Moreover, an \textbf{atomic sentence} is a sentence that an NLP expert cannot decompose into multiple sentences.
The system designed in this project should be general-purpose; changing the dataset on which the concept bottleneck classifier operates should immediately produce relevant concept sentences for a different dataset.

As such, this project aims to propose a novel method, combining techniques from natural language processing, deep learning and logic-based learning, that would extract domain-independent concept sentences.


\section{Limitations and Assumptions}
% Probably talk about in the initial project things.

At the moment, only the syntactic concept generalisation is being made. 
A consequence of using syntactic concept generalisation is that the events happening in the video are isolated as sentences with no content linking between them. 
Additionally, the tokens in a generalised concept sentence must first occur in the original one.
This procedure may result in grammatically incorrect sentences.
For example, consider a sentence: \emph{The left fielder caught the ball.}.
With our current approach, the system can extract the following sentence \emph{The fielder caught the ball.}.
However, the issue is that it can be unclear who \emph{the fielder} is now because it was previously determined by the adjective \emph{left}.

An extension might be able to relax this limitation.
The most straightforward possible approach for resolving this issue may replace \emph{the} with \emph{a} in a generalisation where additional information about the determined noun is removed.

More advanced generalisation could involve swapping words out for their synonyms.
An additional extension could involve linking entities from one sentence to another. 



\section{Objectives}

The objectives of this project are:

 - Develop and evaluate a general syntactic framework for extracting concept sentences from any given sentence. The process would be split into decomposing non-atomic sentences into atomic and generating concept sentences from atomic sentences.
 
 - Explore how does this framework aid deep neural network models in the context of video classification, namely whether an improvement is achieved for the MLB-V2E dataset \cite{RefWorks:RefID:16-2021automatic}.


 - Time permitting, generate a natural language explanation of the label chosen by the classifier. 


\section{Challenges}

This section highlights challenges that are anticipated to happen and a brief overview of its difficulties.
The section will be revised upon completion of the project once the topics' complexity is more evident.

The following questions will be difficult but necessary to answer as a part of this project:

 - \emph{How should a concept extracted from natural language be defined?} Making a helpful concept and immediately extensible to other domains may be difficult. One cannot use expert knowledge to help craft features that the subsequent architecture should use.
 
 
 - \emph{How will the concept extraction pipeline be scaled with a large amount of data?} The proposed system will include logic-based learning systems ILASP \cite{RefWorks:RefID:18-law2020ilasp} to extract syntactic concept sentences. 
 Unfortunately, the system is not scalable with respect to hypothesis space. On the other hand, more scalable alternative FastLAS \cite{RefWorks:RefID:19-law2020fastlas:} includes limitations that may be impossible to workaround.
 
 - \emph{How is an atomic sentence decomposed into a concept sentence?} This is the critical problem the system must resolve to be used.

 - \emph{How should the sentences be decomposed into atomic sentences?} This is one of the critical problems this system would need to resolve to be applied effectively. It may become a much more complex issue than concept sentence extraction, which can be designed as graph pruning of a dependency graph.
 


\section{Contributions}

Not relevant yet.

% What is the problem?
% Why is it interesting?
% What is the main idea for solving it?

